\chapter{Einleitung}


\section{Problemstellung}
Glas hat in der Architektur in den letzten Jahrzehnten eine immer größere Bedeutung erlangt. Die Konstruktionen werden immer filigraner und lichtdurchlässiger. Dieser Trend legt nahe, Glas stärker in die lastabtragende Konstruktion einzubeziehen. Für vertikale Lasten stehen beispielsweise Plattenbalken und Träger zur Verfügung, als aussteifende Elemente können Scheiben verwendet werden.

Durch den Verbund mit Holz soll die negative Materialeigenschaft der vergleichsweise geringen Zugfestigkeit --- bedingt durch die Rissanfälligkeit zufolge Fehlstellen wie Mikroanrissen, Moleküleinschlüssen oder Kerben \cite{Kr04} --- vermindert und die positive Eigenschaft der hohen Druckfestigkeit hervorgehoben werden. Weitere Vor- und Nachteile des Holz-Glas-Verbundbaus und der einzelnen Werkstoffe findet man in den grundlegenden Arbeiten \cite{Kr04} und \cite{Ha99}.

% Durch das Verkleben der beiden Materialien können im Gegensatz zu anderen mechanischen Verbindungen Spitzenspannungen bei der Lasteinleitung in das Glas verringert werden.



\section{Zielsetzung}

Um Holz-Glas-Verbundbauteile effizient anzuwenden ist unter anderem ein zuverlässiges und einfaches Bemessungsverfahren notwendig. Daher sollen in dieser Arbeit für den Anwendungsfall des Plattenbalkens  ausgewählte Berechnungsmodelle und ein semiprobabilistisches Bemessungskonzept vorgestellt und mittels theoretischer und experimenteller Untersuchungen verifiziert werden.

Für den Plattenbalken stehen dafür mehrere Berechnungsverfahren zur Verfügung. Aus dem Holzbau sind vorallem das Gamma-Verfahren und weiters auch das Schubanalogie-Verfahren bekannt. In \cite{Ha99} wird außerdem das Verfahren nach Natterer/Hoeft vorgestellt, das aber hier nicht behandelt wird.
