\documentclass[12 pt,a4 paper ]{scrreprt}
%---------------------------------------------------------- 
\usepackage[utf8]{inputenc}
\usepackage[naustrian]{babel}
\usepackage{lmodern}
\usepackage[T1]{fontenc}
\usepackage{graphicx}
\usepackage{amsmath}
\usepackage{float}
\usepackage{amsmath,amssymb,amstext}
\usepackage{pgfplots}
\usepackage{eurosym}
\usepackage{nicefrac}
\usepackage{units}
\usepackage{color}
%\usepackage[tight]{units}
%\usepackage[loose]{units}
\usepackage{pgfplots, pgfplotstable}
\usepackage[headsepline,plainheadsepline]{scrpage2}
\pagestyle{scrheadings}
\ihead[\rightmark]{\rightmark} \chead[]{}
%\ohead[\pagemark]{\pagemark} \cfoot[]{}
\usepackage{pgfplots, pgfplotstable}
\usepackage{tikz}
\usepackage{pgfplots}

\automark{chapter}
\renewcommand{\chaptermark}[1]{\markright{\ #1}}



\pgfplotsset{width=15cm,compat=1.6}
\usepackage{pgfplots}

 \usepackage[raiselinks]{hyperref}
\usepackage{amsfonts}  
\usepackage{babelbib} 
\usepackage{url} 
%\usepackage[fixlanguage]{babelbib}

\begin{document} 


\begin{figure}

\begin{tikzpicture}
\pgfplotsset{small}
\matrix{
\begin{axis}[
			title = $F_{max}$,
			ylabel= $\lbrack kN\rbrack $,
			ybar,
			enlargelimits=0.25,
			symbolic x coords={BT1,BT2,BT3,BT4},
			xtick=data,
			nodes near coords
			]
\addplot coordinates {(BT1,72) (BT2,30) (BT3,58) (BT4,12)};
\end{axis}

&

\begin{axis}[
			title = $F_{w=l/400}$,
			ylabel= $\lbrack kN\rbrack $ ,
			ybar,
			enlargelimits=0.25,
			symbolic x coords={BT1,BT2,BT3,BT4},
			xtick=data,
			nodes near coords
			]
\pgfplotsset{cycle list shift=1}
\addplot coordinates {(BT1,25) (BT2,15) (BT3,24) (BT4,5.0)};
\end{axis}

\\

\begin{axis}[
			title = $w_{max \left( F=8kN \right) }$,
			ylabel= $\lbrack mm\rbrack $ ,
			ybar,
			enlargelimits=0.25,
			symbolic x coords={BT1,BT2,BT3,BT4},
			xtick=data,
			nodes near coords
			]
\pgfplotsset{cycle list shift=2}
\addplot coordinates {(BT1,5.5) (BT2,9.2) (BT3,6.2) (BT4,58)};
\end{axis}

&

\begin{axis}[
			title = $EI_{equ}$,
			ylabel= $\lbrack MN \cdot m^2\rbrack $ ,
			ybar,
			enlargelimits=0.25,
			symbolic x coords={BT1,BT2,BT3,BT4},
			xtick=data,
			nodes near coords
			]
\pgfplotsset{cycle list shift=3}
\addplot coordinates {(BT1,0.155) (BT2,0.092) (BT3,0.138) (BT4,0.015)};
\end{axis}

\\
};
\end{tikzpicture} 
\caption{Vergleich der Bauteilversuche (Maximallast, Durchbiegung, Biegesteifigkeit)}
\label{abb:vergleich_balkendiagramm}
\end{figure}

\begin{figure}[h!]
\begin{center}
\begin{tikzpicture}
\begin{axis}[height=12cm, width=12cm, 
			no markers,
			xmajorgrids,ymajorgrids,
			xlabel=Verschiebung $\lbrack mm \rbrack $ ,
			xmin=0,ymin=0,
			ylabel=Kraft\,$\lbrack kN \rbrack $,
			legend pos= north east
			]
\addplot [blue,solid] table[y=F1,x=BT1u4]{Auswertung/vergleiche/vergleich_schubverformung1.dat};
\addplot [blue,densely dashed] table[y=F1,x=BT1u5]{Auswertung/vergleiche/vergleich_schubverformung1.dat};
\addplot [red,solid] table[y=F2,x=BT2u4]{Auswertung/vergleiche/vergleich_schubverformung1.dat};
\addplot [red,densely  dashed] table[y=F2,x=BT2u5]{Auswertung/vergleiche/vergleich_schubverformung1.dat};
\addplot [brown!60!black,solid] table[y=F3,x=BT3u4]{Auswertung/vergleiche/vergleich_schubverformung1.dat};
\addplot [brown!60!black,densely  dashed] table[y=F3,x=BT3u5]{Auswertung/vergleiche/vergleich_schubverformung1.dat};
\addplot [black,solid] table[y=F4,x=BT4u4]{Auswertung/vergleiche/vergleich_schubverformung1.dat};
\addplot [black,densely  dashed] table[y=F4,x=BT4u5]{Auswertung/vergleiche/vergleich_schubverformung1.dat};
\legend{BT1\,u4,BT1\,u5,BT2\,u4,BT2\,u5,BT3\,u4,BT3\,u5,BT4\,u4,BT4\,u5}
\end{axis}
\end{tikzpicture}
\caption{Vergleich der horizontalen Verschiebung beim Auflager }
\label{vergleich-schubverformung-u4 und u5}
\end{center}
\end{figure}


\begin{figure}
\begin{center}

\begin{tikzpicture}
\pgfplotsset{small,width=8cm}
\matrix{
 

\begin{axis}[	title = Zusammenhang $EI_{eff}$ und$\gamma$,
				legend pos= south east,
				no markers,
				xmajorgrids,ymajorgrids,
				ymajorgrids,ylabel=$EI_{eff}$\, $\lbrack MN \cdot m^2 \rbrack $ 									,xmin=0,ymin=0,
				xlabel=$\gamma_{1}$\,$\lbrack 1 \rbrack $,
				]
				
\addplot [only marks,color=blue, fill=blue!80!black] coordinates{	(0.58, 13.88)};
\label{pgf:BT1}
\addplot [only marks,color=red, fill=red!80!black] coordinates{	(0.26, 8.19)};
\label{pgf:BT2}
\addplot [only marks,color=brown!60!black, fill=brown!80!black] coordinates{	(0.54, 13.34)};
\label{pgf:BT3}
\addplot [only marks,color=black, fill=black] coordinates{ (0.21,6.98)};
\label{pgf:BT4}
\addplot table[y=EIeff,x=gamma1]{gammaverfahren/abbildungen/werte.dat};				
				
\end{axis} 
 
&

\begin{axis}[	title = Zusammenhang $\gamma$ und $c_{F}$,
				legend pos= south east,
				no markers,
				xmajorgrids,ymajorgrids,
				ymajorgrids,xlabel=$c_F$\,$\lbrack MN/m^2 \rbrack $ 												,xmin=0,ymin=0,
				ylabel=$\gamma_{1}$\,$\lbrack 1 \rbrack $,
				]
\addplot [only marks,color=blue, fill=blue!80!black] coordinates{	(150,0.58)};
\addplot [only marks,color=red, fill=red!80!black] coordinates{	(40, 0.26)};
\addplot [only marks,color=brown!60!black, fill=brown!80!black] coordinates{	(120,0.54)};
\addplot [only marks,color=black, fill=black] coordinates{ (30,0.21)};
\addplot table[y=gamma1,x=c]{gammaverfahren/abbildungen/werte.dat};
\end{axis}


\\
};
\end{tikzpicture}
\fbox{
\ref{pgf:BT1} BT1 \ref{pgf:BT2} BT2  \ref{pgf:BT3} BT3  \ref{pgf:BT4} BT4
 }
 
\caption{Zusammenhänge zwischen $EI_{eff}$ , $\gamma$ und $c_F$)}
\label{abb:zusammmenhaenge}
\end{center}
\end{figure}













\end{document}